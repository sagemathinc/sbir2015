\documentclass[11pt]{article}
\usepackage{amsmath}%
\usepackage{amsfonts}%
\usepackage{amssymb}%
\usepackage{amsthm}%
\usepackage{url}

\hoffset=-.04\textwidth%
\textwidth=1.08\textwidth%
\voffset=-0.03\textheight
\textheight=1.06\textheight
%\voffset=0.5cm%
%\textheight=1.08\textheight%

\newcommand{\C}{\mathbf{C}}%
\newcommand{\F}{\mathbf{F}}%
\newcommand{\Q}{\mathbf{Q}}%
\newcommand{\QQ}{\mathbf{Q}}%
\newcommand{\Qbar}{\overline{\Q}}%
\newcommand{\Z}{\mathbf{Z}}%
\newcommand{\ZZ}{\mathbf{Z}}%
\newcommand{\R}{\mathbf{R}}%
\newcommand{\T}{\mathbf{T}}%
\renewcommand{\H}{\mathrm{H}}%

\newcommand{\eps}{\varepsilon}%
\newcommand{\con}{\equiv}%
\newcommand{\isom}{\cong}%
\newcommand{\rhobar}{\overline{\rho}}
\newcommand{\tensor}{\otimes}


% ---- SHA ----
\DeclareFontEncoding{OT2}{}{} % to enable usage of cyrillic fonts
  \newcommand{\textcyr}[1]{%
    {\fontencoding{OT2}\fontfamily{wncyr}\fontseries{m}\fontshape{n}%
     \selectfont #1}}%
\newcommand{\Sha}{{\mbox{\textcyr{Sh}}}}%
\newcommand{\tor}{\mbox{\scriptsize\rm tor}}

\newcommand{\myname}{William A.\ Stein}
\newcommand{\phone}{}%{\sf (206) 419-0925}}
\newcommand{\email}{}%{\sf wstein@uw.edu}}
\newcommand{\www}{}%\sf \url{http://wstein.org}}}
\newcommand{\address}{}
\usepackage{fancyhdr,ifthen}
\pagestyle{fancy}
\cfoot{\thepage}  % no footers (in pagestyle fancy)
% running left heading
\lhead{\bfseries\LARGE\em \noindent{}\hspace{-.2em}\myname{}
        \hfill \thisdocument\vspace{-.2ex}\\}
% running right heading
%\newcommand{\spc}{1.31em}
\newcommand{\spc}{1em}

%\rhead{\em {\small{\phone{}}} \hfill $\cdot$\hfill
% \email{} \hfill $\cdot$\hfill \www{}}

\rhead{\em {\small{\phone{}}} \hfill \email{} \hfill \www{}}

\setlength{\headheight}{7ex}
\newcommand{\mainhead}[1]{\begin{center}{\Large \bf #1}\end{center}}
\newcommand{\head}[1]{\vspace{1.5ex}\par\noindent{\large \bf #1}\par\noindent}
\newcommand{\subhead}[1]{\vspace{2ex}\par\noindent{\sl #1}\vspace{1ex}\par\noindent{}}
\newcommand{\ptitle}{\sl}

\newcommand{\hra}{\hookrightarrow}


%%%% Theoremstyles
\theoremstyle{plain}
\newtheorem{theorem}{Theorem}[section]
\newtheorem{proposition}[theorem]{Proposition}
\newtheorem{corollary}[theorem]{Corollary}
\newtheorem{claim}[theorem]{Claim}
\newtheorem{lemma}[theorem]{Lemma}
\newtheorem{conjecture}[theorem]{Conjecture}

\theoremstyle{definition}
\newtheorem{definition}[theorem]{Definition}
\newtheorem{algorithm}[theorem]{Algorithm}
\newtheorem{question}[theorem]{Question}
\newtheorem{challenge}[theorem]{Challenge}
\newtheorem{problem}[theorem]{Problem}
\newtheorem{goal}[theorem]{Goal}

\theoremstyle{remark}
\newtheorem{remark}[theorem]{Remark}
\newtheorem{remarks}[theorem]{Remarks}
\newtheorem{example}[theorem]{Example}
\newtheorem{exercise}[theorem]{Exercise}

\DeclareMathOperator{\End}{End}%
\DeclareMathOperator{\Tr}{Tr}%
\DeclareMathOperator{\Res}{Res}%
\DeclareMathOperator{\res}{res}%
\DeclareMathOperator{\BSD}{BSD}%
\DeclareMathOperator{\Gal}{Gal}%
\DeclareMathOperator{\GL}{GL}%
\DeclareMathOperator{\Aut}{Aut}%
\DeclareMathOperator{\Reg}{Reg}%
\DeclareMathOperator{\Vis}{Vis}%
\DeclareMathOperator{\Ker}{Ker}%
\DeclareMathOperator{\Coker}{Coker}%
\DeclareMathOperator{\Sel}{Sel}%
\DeclareMathOperator{\ord}{ord}%
\DeclareMathOperator{\new}{new}%
\DeclareMathOperator{\an}{an}%
\DeclareMathOperator{\Vol}{Vol}%

\newcommand{\ran}{r_{\an}}

\DeclareMathOperator{\rank}{rank}%
\DeclareMathOperator{\Div}{Div}
\DeclareMathOperator{\sss}{ss}
\renewcommand{\ss}{\sss}
\renewcommand{\O}{\mathcal{O}}

\DeclareMathOperator{\Frob}{Frob}
\newcommand{\cA}{\mathcal{A}}
\newcommand{\cN}{\mathcal{N}}
\newcommand{\frakp}{\mathfrak{p}}

\DeclareMathOperator{\Disc}{Disc}
\DeclareMathOperator{\cond}{cond}
\newcommand{\thisdocument}{Biographical Sketch}

\begin{document}
%\maketitle

\head{Professional preparation}%
\begin{center}
\begin{tabular}{lll}
  % after \\: \hline or \cline{col1-col2} \cline{col3-col4} ...
\mbox{}\hspace{3.2ex}
&  Harvard University & NSF Postdoc, 2000--2004 \\
&  University of California at Berkeley & Mathematics, Ph.D. 2000 \\
&   Northern Arizona University\hspace{1.03in}\mbox{}& Mathematics, B.S. 1994 \\
\end{tabular}
\end{center}

\head{Appointments}
\begin{itemize}\setlength{\itemsep}{-0.8ex}
\item {\em Professor} of Mathematics,
University of Washington, September 2010--now.
\item {\em Associate Professor} of Mathematics,
University of Washington, April 2006--August 2010.
\item {\em Associate Professor} of Mathematics,
UC San Diego, July 2005--March 2006.
\item Benjamin Peirce {\em Assistant  Professor}  of Mathematics,
Harvard University, July 2001--May 2005.
\item {\em NSF Postdoctoral} Research Fellowship
under Barry Mazur at Harvard University, August 2000--May 2004.
\item Clay Mathematics Institute Liftoff Fellow, Summer 2000.
\end{itemize}


\head{Products related to proposal}%
\begin{itemize}\setlength{\itemsep}{-0.8ex}

\item Founded SageMathCloud in 2013 (see \url{https://cloud.sagemath.com}).

\item Founded SageMath in 2004
  (see \url{http://sagemath.org}), which is a large free open source software
  project that has over 50{,}000 active users.
\item \emph{Elementary Number Theory: Primes, Congruences, and Secrets} (185 pages), published in the Springer-Verlag UTM series, 2008.

\item \emph{PRIMES} (136 pages), with B. Mazur, a
book on the Riemann Hypothesis, under contract with Cambridge Univ. Press (see \url{http://wstein.org/rh/}).

\item \emph{The Sage Project: Unifying Free Mathematical Software to Create a Viable Alternative to Magma, Maple, Mathematica and Matlab}
(2010), with B. Erocal, for plenary talk at the 2010 International Congress of Math. Software.

\end{itemize}

\newpage
\head{Other significant products}%
\begin{itemize}\setlength{\itemsep}{-0.8ex}

\item {\ptitle Non-commutative Iwasawa theory for modular forms} (40 pages), with
J.~Coates, T.~Dokchitser, Z.~Liang, R.~Sujatha, 2013, in Proceedings of the LMS.

\item {\ptitle Computations About Tate-Shafarevich Groups Using
    Iwasawa Theory} (46 pages), with C.~Wuthrich, 2012, Mathematics of Computation.

\item {\ptitle Heegner Points and the Arithmetic of Elliptic Curves
    over Ring Class Extensions}, with R. Bradshaw (15 pages), 2012, J. Number Theory.

\item {\ptitle Toward a Generalization of the Gross-Zagier Conjecture}
  (33 pages), Int Math Res Notices (2011) Vol. 2011 309-341.

\item \emph{Modular forms, a computational approach} (xvi+268 pp.)
  Graduate Studies in Mathematics (AMS) 79 2007, with an appendix by
  Paul Gunnells.

%\item {\ptitle Shafarevich-Tate Groups of Nonsquare Order}, Progress
%  in Math., {\bf 224} (2004), 277--289, Birkhauser.



\end{itemize}

\head{Synergistic activities}
\begin{itemize}\setlength{\itemsep}{-0.5ex}
\item ACM/SIGSAM 2013
winner of the {\em Richard Dimick Jenks
Memorial Prize} for Excellence in
Software Engineering applied to Computer Algebra.

\item SIMUW 2006, 2007, 2008, 2012; Canada/USA MathCamp
  mentor (2002); Math Circles talks in Boston; 2011 REU on elliptic
  curves; 2013 REU on Sage; involved dozens of undergraduates in work on the Sage software.

\end{itemize}

\head{Collaborators and other affiliations}
\newcommand{\hl}[1]{#1}%
\begin{itemize}\setlength{\itemsep}{-0.5ex}
\item \textbf{Collaborators during last 48 months (23 total):}  Jennifer S. Balakrishnan, Jonathan Bober, Robert Bradshaw, Mirela Ciperiani, John Coates, Henri Darmon, Michael Daub, Alyson Deines, Tim Dokchitser, Burcin Erocal, Ariah Klages-Mundt, Benjamin LeVeque, Zhibin Liang, Sam Lichtenstein, J. Steffen M\"{u}ller, Barry Mazur, R. Andrew Ohana, Clement Pernet, A. Rabindranath, Victor Rotger, Paul Sharaba, Ramdorai Sujatha, Christian Wuthrich


\item \textbf{Graduate Advisors and Postdoctoral Sponsors (2 total):}\vspace{-1ex}
\begin{itemize}\setlength{\itemsep}{-0.5ex}
\item {\bf Ph.D. advisor:} Hendrik Lenstra, University of Leiden,
Netherlands.%
\item {\bf NSF Postdoctoral advisor:} Barry Mazur, Harvard
University.
\end{itemize}
\item \textbf{Thesis Advisor and Postgraduate-Scholar Sponsor (8 total):}
\begin{itemize}\setlength{\itemsep}{-0.5ex}
\item  6 Ph.D. students:
Robert Bradshaw (Google, 2010 Ph.D.); Robert Miller's(Applauze, 2010 Ph.D.); Alyson Dienes (CCR, 2014 Ph.D.);
Simon Spicer's (Ph.D., June 2015);
Hao Chen's (Ph.D. expected June 2016);
Andrew Ohana (Ph.D. expected June 2016);
\item 2 Postdocs: Clement Pernet (Grenoble; postdoc 2007--2008), Craig Citro (Google; postdoc 2009).
\end{itemize}
\end{itemize}

% \newpage

% \begin{center}
% \head{Biographical Statement}
% \end{center}

% \noindent{}William Stein will contribute to this project in both a
% research and managerial role.  Stein has been a driving force over the
% last 10 years in applications of computation to research on modular
% forms, $L$-functions, and associated arithmetic objects.  As director
% of the Sage project (http://sagemath.org), he has experience managing
% working groups and working with undergraduates on a wide range of
% projects.

% \vspace{2ex}


% \noindent{}{\bf Support Statement:}
% The proposed project naturally fits in with my other
% NSF-funded research on the Birch and Swinnerton-Dyer conjecture, from
% which I will receive 2 months summer support during the next 2 years
% and funding for travel and materials (DMS-0653968).  I have also
% received an NSF grant (DMS-0703583) to support one postdoc for three
% years, who will work on developing exact linear algebra algorithms and
% implementations for Sage.  I am a co-PI on the Arizona Winter School
% grant (DMS-0602287); this is a yearly 1-week 120-person graduate
% student workshop in arithmetic geometry, whose upcoming topics mesh
% well with the themes of the current proposal (e.g., the next theme is
% quadratic forms, and theta series of quadratic forms are modular
% forms).  I will also be directing graduate and undergraduate research
% that is related to this project during the academic year, not just
% during the summer.  Finally, I am also applying to the NSF FRG program
% for additional money to support a postdoc and workshops.

\end{document}

%sagemathcloud={"zoom_width":130}